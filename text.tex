\documentclass[10pt,a4paper,notitlepage]{report}
\usepackage[utf8]{inputenc}
\usepackage[czech]{babel}
\usepackage[T1]{fontenc}
\usepackage{amsmath}
\usepackage{amsfonts}
\usepackage{amssymb}
\usepackage{makeidx}
\usepackage{graphicx}
\usepackage{syntax}    % texlive-mdwtools
\usepackage{glossaries}
\usepackage{todonotes} % command \todo
\usepackage{hyperref}
\usepackage{listings}
\author{Jan Tušil <410062@mail.muni.cz>}
\title{LLVM PO}
\begin{document}

\chapter{Cíl práce}
Exploze stavového prostoru při model checkingu, Partial Order Redukce to řeší,
LLVM model checking je super a chceme tam POR.

To, co chceme, je vygenerovaný NTS, který není paralelní. Tedy výstupem redukce
není explicitní stavový prostor.

\chapter{Použité technologie}
Proč volíme právě NTS? Jakou podmnožinu těch jazyků podporuji?
\section{LLVM}
\subsection{Typový systém}
\label{subsec:llvm-type-system}
\subsection{Omezení}
Nestaráme se o pointery, nepodporuji pole (zatím)


\section{NTS}

\subsection{Popis originální verze}
Potřeba říci, že máme datové typy Int, Bool, Real. Int, Real jsou matematické, s neomezeným rozsahem a (u Real) s neomezenou přesností.

\subsection{Rozšíření}

Jak můžeme vidět v předchozích částech, typový systém jazyka NTS se od typového systému LLVM IR 
%\ref{subsec:llvm-type-system}
v mnohém liší. Zatímco v LLVM IR mohou stejný datový typ (Integer type, například i8) využívat jak aritmetické instrukce (binary instructions), tak bitové logické instrukce (bitwise binary instructions), jazyk NTS již na úrovni syntaxe podporuje logické operace pouze nad termy typu Bool. Pro překlad LLVM IR do NTS je tedy nezbytné cílový jazyk vhodným způsobem rozšířit. Možností je více:
\begin{enumerate}
\item Rozšířit logcké operace i nad datový typ Int. Jaký by ale byl význam například negace? Totiž, pro různou bitovou šířku dává negace různý výsledek. Budeme-li mít například proměnnou typu Int s hodnotou 5 (binárně 101), pokud se na ní budeme dívat jako na 4 bitové číslo (tedy 0101), dostaneme po negaci 10 (1010), pokud se na ní budeme dívat jako na 3 bitové číslo, dostaneme po negaci hodnotu 2 (binárně 010). 

\item Zavést speciální operátory, které budou v sobě obsahovat informaci o uvažované bitové šířce.

\item Zavést datový typ BitVector<k>, který je vlastně k-ticí typu Bool. Na něm můžeme dělat logické operace bitově a aritmetické operace jako na binárně reprezentovaném čísle. 
\end{enumerate}

Volíme třetí způsob.

V syntaxi původního nts je rozlišováno mezi aritmetickým a booleovském literálem. Chtěli bychom nějak zapisovat bitvectorové hodnoty. Ve hře jsou dvě možnosti:

\begin{enumerate}

\item Pro zápis konstant typu BitVector<k> bychom mohli používat speciální syntaxi. Například zápis 32x"01abcd42" může představovat konstantu typu BitVector<32>, zapsanou v hexadecimálním tvaru. Toto řešení je navíc vhodné pro zápis speciálních konstant (jako $2^{31}$) v lidsky čitelném tvaru. 

\item Aritmetické literály mohou mít schopnost být Int-em nebo BitVector<k>-em podle potřeby (polymorfismus?).

\end{enumerate}

Použita byla druhá možnost (také si jí tu podrobněji popíšeme). V případě potřeby můžeme do jazyka přidat i tu první, ale už ne kvůli potřebě rozlišovat mezi datovými typy, ale kvůli užitečnosti zapisovat některé konstanty hezky.

Syntaxi jazyka zjednodušíme tak, že definujeme jeden neterminál <term> místo dvou neterminálů <arith-term> a <bool-term>. S každým termem bude ale spojená sémantická informace, kterou je jeho datový typ. Definujeme následující skalární datové typy:

\begin{enumerate}
\item Int
\item BitVector<k>, $\forall k \in \mathbb{N^+}$
\item Real
\end{enumerate}

Původní datový typ Bool chápeme jako zkratku za typ BitVector<1>

\begin{grammar}
<literal> ::= <id> | tid | <numeral> | <decimal>

<aop> = `+'  | `-' | `*' | `/' |  `\%'

<bop> = `&' | `|' | `->' | `<->'

<op> = <aop> | <bop>

<term> ::= <literal> | <term> <aop> <term> | <term> <bop> <term> | `(' <term> `)'
\end{grammar}
%TODO mozna by v gramatice melo byt:
% <uop> = `-' | "not"
% <term> ::= .... <uop> <term>
% to ale neni 100% podporovane. Nebo ano?
%TODO gramatika muze byt v plovoucim prostredi

\subsubsection{Typovací pravidla}
Abychom zajistili, že syntakticky stejný výraz může být typu BitVector nebo Int, definujeme typovou třídu Integral se členy BitVector<k> a Int. Typ výrazu je definován rekurzivně

\begin{enumerate}
\item[TR1] Numerická konstanta <numeral> je libovolného typu 'a' z třídy Integral.
\item[TR2] <decimal> je typu Real
\item[TR3] tid je libovolného typu 'a' z třídy Integral
\item[TR4] <id> je stejného typu, jako odpovídající proměnná
\end{enumerate}

%BTW BitVector můžeme chápat jako typový konstruktor
%TODO neni korektni delat logicke operace nad dvemi nespecifikovanymi integraly.
%V pravidlech to zachytim, ale v programu to zatim neni.

Mějme termy\\
a1 :: Integral a => a\\
a2 :: Integral b => b\\
b1 :: BitVector<k1> \\
b2 :: BitVector<k2>\\
i1 :: Int\\
i2 :: Int\\

Potom:
a1 <aop> a2 :: Integral a => a
a1 <op>  b1 :: Bitvector<k1>
a1 <aop> i1 :: Int
b1 <op>  b2 :: BitVector<max{k1,k2}>
i1 <aop> i2 :: Int

Tato pravidla platí i komutativně. Co se do nich nevejde, není typově správný výraz.

Tedy přidáváme datový typ BitVector, typové třídy a implicitní typová konverze.

Btw můžeme využívat anotace. Zatím je využíváme jenom trochu, ale dají se o nich vkládat
i nějaké informace z LLVM nebo další postřehy.

\subsection{Omezení}
Zatím nepodporuji složitější operace s poli. Také nechápu, jak pracovat s parametry (par).

\subsubsection{Pole}

\subsection{Možná budoucí rozšíření}
\subsubsection{Znaménkové datové typy}

\chapter{Jak na to?}

\section{Použitá terminologie}

\section{Velký obraz}

%Velký obraz - jak se vypořádat s funkčními voláními? Jaké jsem měl možnosti?

%Výsledkem trojice: llvm2nts, inliner, vlastní POR. Všechno pracuje nad knihovnou pro paměťovou
%reprezentaci NTS. Related work: Petrův parser.

%Všechno ve formě knihovny - jednoduché rozhraní, snadno použitelné.

%BTW: NTS je 'ten' - system
\subsection{Funkční volání} %trochu nevhodný název pro sekci "Velký obraz"

Chceme tedy z paralelního NTS, který vznikl překladem z LLVM IR, učinit sekvenční. Pokud o vstupním paralelním NTS nebudeme nic předpokládat, nástroj pro sekvencializaci bude muset umět korektně zacházet s funkčními voláními, protože ta mohou být součástí NTS. To zejména znamená, že nástroj si bude muset pro každý stav výstupního systému udržovat přehled o tom, jaké funkce jsou v jakém vlákně aktivní. Implementace takového nástroje nemusí být snadná, proto jsme učinili následující rozhodnutí:

% tohle bych chtel v sedem ramecku, podobne jako byvaji citace nebo theoremy
Nástroj pro sekvencializaci předpokládá, že vstupní paralelní systém neobsahuje funkční volání, tedy (v termínech) NTS není hierarchický. Takové NTS budeme dále označovat jako "ploché".

Většina skutečných programů v jazyce LLVM IR ovšem obsahují funkční volání, a rádi bychom takové programy podporovali. Nabízejí se dvě možnosti, jak takovou podporu zajistit. První spočívá v tom, že paralelní program v jazyce LLVM IR přeložíme na paralelní a (potenciálně) hierarchický přechodový systém v jazyce NTS, který následně převedeme na ekvivalentní paralelní plochý přechodový systém. Alternativní možností je nejprve odstranit funkční volání z paralelního LLVM IR programu a následně tento plochý program převést na plochý paralelní přechodový systém.

Tak či onak, programy s rekurzivním voláním nebudou podporovány.

\paragraph{Zplošťování LLVM}
Zploštění programu v LLVM IR nevyžaduje téměř žádnou další práci, protože již na tuto úlohu existuje llvm průchod. Jmenuje se
\href{http://llvm.org/docs/Passes.html#inline-function-integration-inlining}
{inline} a je součástí projektu LLVM. Nicméně nástroj pro překlad LLVM IR na NTS musí umět korektně přeložit volání funkce
\href{http://man7.org/linux/man-pages/man3/pthread_create.3.html}
{pthread_create(3)}, které v tomto případě nemůže být přeloženo jako volání BasicNts na úrovni NTS. Volání této funkce by tedy muselo být vhodně přeloženo již na úrovni LLVM kódu.

\paragraph{Zplošťování NTS}
Přestože je pro zplošťování NTS potřeba udělat o něco více práce, napsání odpovídajícího nástroje nám může umožnit sekvencializaci libovolného NTS. Navíc nejsme omezeni na práci s plochými NTS jako v předchozím případě. Z uvedených důvodů tato práce využívá tento přístup.

\subsection{Architektura}
Celý problém se tedy rozpadá na několik částí. Nejprve potřebujeme přeložit program z jazyka LLVM IR do formalizmu NTS, poté z přeloženého programu odstranit volání subsystémů, plochý program sekvencializovat s využitím partial order redukce a nakonec ho uložit do souboru. Pro spolupráci těchto částí je třeba mít vybudovanou vhodnou paměťovou reprezentaci programu v jazyce NTS. Existence parseru by byla užitečná (zejména pro testování), ale není nutná.

%Přestože existuje


\section{Architektura libNTS}
- inspirováno LLVM


\section{Překlad llvm na nts}
Btw nepatří rozhodnutí o omezení vstupního jazyka sem?

\subsection{Model paralelismu}
\todo{Možná, že popis paralelních modelů LLVM a NTS nepatří do této sekce. Někde to ale vysvětlit musím.}

\subsubsection{LLVM, pthreads}
Jazyk LLVM IR sám neposkytuje podporu pro explicitní paralelismus. Proto podporujeme posixovou knihovnu pro vlákna, pthreads (resp. její část). Nejzajímavější funkcí z této knihovny je pthread\_create, která jako jeden ze svých argumentů přijímá pointer na funkci, jež následně spustí v novém vytvořeném vlákně.Volání této funkce se může ve zdrojovém programu vyskytnout kdekoliv, a na umístění a četnost těchto výskytů neklademe žádná omezení. Omezujeme ale možnosti použití této funkce, a to následujícím způsobem:

\begin{enumerate}
\item Argument parametru start\_routine musí být přímo funkce, nikoliv jiný funkční ukazatel.
\item Všechny ostatní argumenty funkce pthread\_create musí být null.
\item Návratová hodnota funkce pthread\_create nesmí být použita.
\item Funkce, předaná jako argument parametru start\_routine, nesmí používat svůj parametr.
\item Návratová hodnota této funkce musí být vždy null.
\end{enumerate}

\begin{figure}[h!]
\begin{lstlisting}[language=C]
int pthread_create ( pthread_t *thread,
                     const pthread_attr_t *attr,
                     void *(*start_routine) (void *),
                     void *arg                        );
\end{lstlisting}
\end{figure}

\subsubsection{NTS}
Jazyk NTS řeší paralelismus odlišným způsobem. Počet vláken je po celou dobu běhu programu stejný, tedy každé vlákno běží již od začátku. Každý paralelní NTS obsahuje řádek \todo{ne nutně řádek} ve tvaru
\begin{lstlisting}
instances basic_nts_1[2], basic_nts_2[3*N+4], ... , main_nts[1];
\end{lstlisting}
který určuje, jaké vlákno bude vykonávat jaký přechodový (sub)systém (BasicNts). Uvedený příklad specifikuje, že dvě vlákna (s id 0 a 1) budou vykonávat BasicNts basic\_nts\_1, dalších několik vláken (s id 2, 3, 4, \ldots 3*N+5) bude vykonávat basic\_nts\_2 a vlákno s nejvyšším id bude vykonávat main\_nts.

\subsubsection{Překlad}
V případě paralelního LLVM překládací program vygeneruje NTS, jež vytváří 1 hlavní vlákno a N vláken pracovních. Každé pracovní vlákno má přiřazeno stav a v každém čase se nachází buď ve stavu nečinném, nebo ve stavu činném. Protože na začátku běhu programu v jazyce LLVM IR je spuštěno pouze jedno vlákno, zatímco v přechodovém systému NTS jsou spuštěna všechna vlákna, chceme, aby běžící pracovní vlákna neměla žádný \textit{efekt}. Tedy pracovní vlákno bude ve výchozím stavu v režimu nečinném. Veškerá volání funkcí \texttt{pthread\_create} nahradíme kódem, který způsobí, že se nějaké nečinné pracovní vlákno stane činným a začne vykonávat kód odpovídající funkce. Po ukončení běhu této funkce vlákno opět přejde do stavu nečinnosti a je tak možné ho znovu využít.



\section{Inlining}
Jo teda nepodporuju rekurzi, a proto si můžu dovolit to, co dělám v inlineru - prostě tak dlouho zainlinovávám jednotlivé BasicNts, až mi nezůstané žádné volání.

\section{POR}
Pozor, POR je víc druhů. Uvézt chytrou knížku, Ample sety.
\subsection{Jak to má fungovat}
Nakonec jde jenom o to, zda proměnnou, kterou nějaký přechod používá, používá i jiné vlákno
\subsection{Problém velkého procesu}
Protože každé vlákno (kromě vlastního) může potenciálně vykonávat libovolnou úlohu, tak téměř každé vlákno může použít téměř každou proměnnou. Redukce by se zredukovala na pouhý test "používám globální proměnné"?
Tedy je potřeba mít rozdělené stavy / přechody do úloh. O každé úloze spočítáme, jaké globální proměnné používá, a také, jaké jiné úlohy může aktivovat. Potom, pokud budeme znát řídící stav každého vlákna, můžeme zjistit, jaké úlohy běží a tedy i jaké globální proměnné jsou důležité.

\subsection{Problém velkého pole}
V případě, že bychom měli pole takové, že by na každou jeho pozici přistupoval nejvýše jeden proces,
a procesů bychom měli mnoho, vyplatilo by se sledovat jeho jednotlivé buňky zvlášť. To ale neděláme. Btw jedno takové pole máme.

\subsection{Problém závislosti na datech}
Zda může nějaké vlákno běžet, závisí na datech. My se ale o data moc nestaráme (TODO: tohle je třeba ujasnit na začátku). Tedy nemůžeme vědět, že na začátku poběží jenom hlavní vlákno. Tedy zeserializovaný systém bude obsahovat běhy, jejihž podmínka cesty bude nesplnitelná. Obecně tohle řešit snadno nelze, ale pokud se omezíme na zjištění informace, zda nějaké vlákno z thread poolu může začít vykonávat nějakou úlohu, stačí nám sledovat pár zvolených proměnných. Na to máme dvě možnosti:

a) Analyzovat vykonávané přechody, zda modifikují naše vybrané proměnné. Předpokládáme, že většina formulí bude mít hezký tvar, a že tedy nemusíme vědět všechno na to, abychom některé mohli rovnou označit za nesplnitelné a o jiných prohlásit, že modifikují námi vybranou proměnnou jednoduchým způsobem.

b) Umět rozpoznat původně existující struktury i v přeloženém a zplacatělém přechodovém systému. Tedy musíme vědět, co jsou pracovní vlákna, co dělá \_\_thread\_create (a jak jí poznám) a další věci.





\end{document}